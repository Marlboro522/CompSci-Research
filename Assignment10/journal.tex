\documentclass{article}[10pt]
\usepackage{geometry}[margin=2cm]
\title{Journal\#10}
\author{Raja Kantheti}
\date{\today}
\begin{document}
\maketitle
\section*{Progress and what I need to do on the paper:}
\begin{itemize}
    \item Only the experimentation part is pending on the paper. I have my experiment set up and now I am dealing with minor issues which can be managed by the deaddline. 
    \item I need to write the results and discussion part of the paper.
    \item I need to write the conclusion and future work part of the paper.
    \item Interpreting the results should be easy as I have already done the same for several projects through out my Master's.
\end{itemize}
\section*{Summary of my learning experience: }
My experience with the paper-writing process has been a gradual journey of turning scattered ideas and discoveries into a structured and meaningful narrative. It began with a thorough search for relevant papers, learning to navigate digital libraries, and refining keyword searches to find sources that directly addressed my research questions. As I collected these papers, I learned how to skim abstracts, identify gaps in the literature, and highlight methods or results that informed my own work. This initial stage helped me understand the broader context of my research and pinpoint the unique contribution I wanted to make.

Once I had a strong collection of references, I moved on to organizing them into a coherent story. I discovered the importance of grouping them based on themes, methods, or conceptual frameworks, and linking these groups logically so that my narrative flowed smoothly from background and motivation through challenges and solutions. While drafting, I found that rewriting and restructuring were not signs of failure but rather essential parts of shaping a clear, persuasive argument. Each revision taught me to place the most critical points in the spotlight and to ensure that the transitions between ideas were seamless and intuitive for readers.

Finally, incorporating feedback proved invaluable. the feedback from colleagues, it often highlighted areas where my explanations were unclear, assumptions went unverified, or evidence was insufficient. Learning to embrace these suggestions and refine my work further improved both the scientific rigor and clarity of my writing. In the end, the process of finding literature, reviewing it, crafting a coherent narrative, and responding to feedback transformed a rough collection of insights into a paper which I feel comfortable enough to read. 
\section*{What I could've done better and what I did bad:}
I could've done the foundational aspects of writing a paper better. I could've been more brutal on the reading and making notes. Another aspect is that I could have planned my time better, but this should not have been a problem for my second paper. I did my best in the first draft. I felt exhilarated after I refined my ideas into a proper first draft. I will continue to be at the same level of excitement whenever I set to write more papers in the future.
\end{document}