\documentclass[10pt]{article}
\usepackage{geometry}[margin=2cm]
\title{Journal\#11}
\author{Raja Kantheti}
\begin{document}
\maketitle
\section{A Brutal Introspection}
\subsection*{Aspects of Research:}
Before signing up for the course, I knew it would be terribly hard for me. The course required a lot of reading and writing and would test my ability to convey my insights and not-so-original ideas. I wouldn't say I was surprised at how much focus I had to put in, but I have to say that going through it was tough for my constantly distracted brain.

I noticed several things in this journey; it takes me three times more effort, time, and focus to comprehend what, for example, one paper in my literature review is conveying. I say I only realize this now because I can quickly grasp, intuitively, any level of complicated topic in the computer science field when I practically do it or while I am listening to a lecture with visual and auditory stimuli (It had been the same level of difficulty when I tried to listen to papers). I felt mentally drained for about a month when I tried to comprehend the literature collection for the survey paper sufficiently correct. This might sound like a common problem for anyone reading research papers for the first time. However, because of my pure passion for the field, I have cultivated this habit for a very long time, no matter how hard it was. Yet I couldn't get through the reading part of Computer Science in academic sense or in any other aspect of my life in any sense.

Writing, on the other hand, has been a different journey. While I may face challenges with the writing aspect of Final Survey Paper, I am confident that I can improve as I continue to write more in the future. I have a plan for this.

Although I am passionate about contributing to the field of computer science, I am saddened and disappointed in myself to know that I cannot pursue this dream at this time. I have a lot more obstacles to overcome and a lot more discomfort to deal with. I shall strive harder in life and career to make this dream a reality.

\subsection*{Time Management:}
While my time management has improved compared to past experiences, I still feel there's room for growth. This course helped me realize the steps I’ve taken towards becoming more productive, but I know I haven’t reached the level I aspire to yet. Even so, every small improvement is a meaningful milestone that encourages me to keep striving forward.

\subsection*{Generic Mediocrity: }
In these ending days of this course, I concluded that I need to reiterate my definition of research. I used to read random research papers with interesting acronyms to get the intellectual high off the knowledge they presented (It usually took weeks to read the papers). As I narrowed down to one problem I wanted to focus on, everything seemed to be very complex, and at one point, I became clueless about what I wanted to do. I will see to it that I will never be this clueless again by exploring more efficient reading techniques. I could have been proactive and had anticipated these problems before signing up for the course, which I usually do for other classes. I should've been more aware and diligent and created a mental environment where I could've been better than this. But these moments pushed me to become more intentional about my growth. I acknowledge that I didn't always meet my expectations, and I will strive to be warier in the future.

\section{Most significant Learning Experience}
\subsection*{Reading: }
I became more aware than ever of my capabilities in reading. This course gave me a chance to know my current limitations, and I will continue to improve in this area in the future. Making notes about what the paper presented helped me overcome half of my problems. This, in turn, helped me get some ideas and write my own insights.

\subsection*{Writing: }
Throughout the course, I experienced significant enhancements in my ability to articulate complex ideas with clarity and precision. The structured assignments and feedback mechanisms provided a support system that challenged me to refine my narrative skills. I learned to organize my thoughts more logically, ensuring each paragraph contributed clearly to the overall argument. This was not just about aligning sentences; it was about weaving a coherent story that resonated with and engaged my readers. I will focus on this for my Final Survey Paper.

\subsection*{The Art of Storytelling: }
I felt empowered as I realized the importance of crafting a narrative that resonates precisely with the audience. I never knew such a concept existed until I encountered Dr.Kendall Haven's perspectives. His talk revealed how a narrative can profoundly transform, shape perceptions, and influence understanding. I read some of your papers and saw your diligence in handling what the reader hears. It changed my way of articulating sentences when I talk to people; I hope to be this aware and careful someday while writing, too. This period is the highlight of the whole course for me.

\subsection*{Journaling:}
I journaled my whole life to process emotions and hardships. The experience of journaling progress, insights, and realizations about what I am doing was enjoyable, just like talking to myself.

\section{Least Significant Learning Experience}
The course never had a non-significant effect on me at any point. It was very well designed, and I learned a lot about my research abilities and limitations from it. It was a very fulfilling and pleasant experience for me.

\section{Coda: Reflective Respite}
As this course in Introduction to Computer Science draws to a close, I pause to reflect on an intensely personal and academically demanding journey. Each lecture, each assignment, and every painstaking hour of reading was not merely an academic task; they were endeared moments of self-discovery and intellectual challenge.

The difficulties, mediocrity, and subsequent mental exhaustion were not mere setbacks but voices of past challenges, each marking a significant step in my ongoing journey of learning and self-improvement. My journals captured not just a record of academic progression but a deeper, more personal evolution, a candid testament to the internal battles fought and the insights gained.

Dr.Kendall Haven's insights on the art of storytelling have profoundly shaped my understanding of how narratives influence our understanding of complex material. This course has underscored the role of compelling communication in scientific writing, reflecting the critical need to articulate complex ideas with precision and clarity.

I am deeply grateful for this course, not only for the skills it has cultivated but for reinforcing the belief that conquest of adversity is a continuous aspect of life. I leave this course not just as a student who has engaged with scientific research but as a person who is perpetually learning to overcome, and adapt. I hope to be better than I am at present in computer science research.

\bigskip

\begin{center}
    \noindent\textbf{Thank you for the extraordinary class, Dr.Boult.}
\end{center}


\end{document}