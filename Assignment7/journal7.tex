\documentclass[14pt]{extarticle}
\usepackage[margin=2cm]{geometry}
\renewcommand{\familydefault}{\sfdefault}
\title{Journal\#7}
\author{Raja Kantheti}
\date{\today}
\begin{document}
\maketitle
\section*{Learnings on the first draft: }
\begin{itemize}
    \item I was able to regain my story and the flow of the paper.
    \item I became more conscious of the flow of the paper and the story that I am trying to tell.
    \item I was able to write the paper in a more structured way.
    \item I also became more fluid in organizing the knowledge I get while I read multiple papers.
    \item I also find myself mimicking the writing style of the papers I read, I am not really sure whether it is a good thing.
\end{itemize}
\section*{Learnings on reviewing: }
\begin{itemize}
    \item It was fascinating to see how people visualize their ideas.
    \item I found that the articuation of the ideas is very important. I had resonance with their way of expression of ideas more than I have on my paper. 
    \item 
\end{itemize}

\section*{Learnings on the experimental design: }
Fallacy\#4 was an eye opener. I was always in this cycle that demonstration is the end of theory.
I am yet to see if there are any other metrics I can rely on to explain my story better, but for now I am skeptical whether the metrics i chose precisely convey my story.
Along my aspirations to enroll into Ph.D program, I think the lecture would be useful to me in drafting a proposal.

The segment about DARPA program evaluation was very interesting. I can see how robust explanations are far better, in providing insight into the depth of a theory such as interaction effects, than mere demonstrations.
I also learned that statiscal significance is not the end of the story, the relevance of the practical impact is more important.
\end{document}