\documentclass[10pt]{article}
\usepackage[margins=2cm]{geometry}
\title{Journal 8}
\date{Due 3/23/18}
\author{Raja Katheti}
\begin{document}
\maketitle
\section{Learnings from peer reviews}
\begin{itemize}
    \item I was able to apply the feedback and get the paper to resonante with the story I want to tell. 
    \item I was able to get the paper to be more concise and to the point.
    \item I was able to add some more details that would enhance each technique and also corrected the references.
    \item I am looking into adding more inferences from the papers I read so that I can discusss them at length when appropriate. 
\end{itemize}
\section{Experimental CS:}
I explain below how I want to able to incorporate the learnings from the class into my experimental setup.
\begin{itemize}
    \item I got to learn what metrics to chooose to align with the story. 
    \item I want to be able to make meaningful deductions from those metrics.
\end{itemize}

Here is my current experimental setup:
\begin{table}[ht]
    \centering
    \caption{Experimental Setup and Metrics}
    \begin{tabular}{|p{4cm}|p{10cm}|} % chktex 44
        \hline % chktex 44
        \textbf{Component} & \textbf{Description} \\
        \hline % chktex 44
        Simulator & gem5 \\
        \hline % chktex 44
        Architecture & RISC-V core with out-of-order execution \\
        \hline % chktex 44
        Processor Model & Clock Speed: 2.5GHz, Cache Configuration: yet to decide, Branch Predictors: Local, BiMode, Tournament, L-TAGE, Multi-Perspective Perceptron \\
        \hline % chktex 44
        Benchmark Suite & SPEC CPU \\
        \hline % chktex 44
        Benchmarks Used & INT and FLOAT \\
        \hline % chktex 44
        Metrics & Prediction Accuracy, Misprediction Rate, Performance Improvement (IPC), Execution Time, Hardware Overhead (Area and Power) \\
        \hline % chktex 44
    \end{tabular}
\end{table}
I am looking into adding more metrics and decide on the cache configuaration where the experiment won't be biased on the spatial locality feature of the cache. 
\end{document}