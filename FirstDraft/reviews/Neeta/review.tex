\documentclass[12pt]{article}
\usepackage{geometry}
\usepackage{graphicx}
\usepackage{array}
\usepackage{caption}

\title{Peer Review}
\author{}
\date{}

\begin{document}
\maketitle

\paragraph{}The paper has a limited amount of material for review. Most sections were inadequate, and no references were cited throughout the paper to provide context for what the author wanted to convey. 

\subsection*{Positive Aspects}
\begin{itemize}
    \item \textbf{Comprehensive Coverage}: The paper provides an adequate overview of how large language models (LLMs) are applied in various bioinformatics tasks, such as gene sequence analysis, protein function prediction, and drug discovery. This breadth of coverage is commendable as it gives readers a holistic view of the current state of research in this area.
\end{itemize}

\subsection*{Areas for Improvement}
\begin{itemize}
    \item \textbf{Clear Identification of Challenges and Opportunities}:T he sections de-
    tailing the key challenges (e.g., model bias and data representation challenges)
    and opportunities (e.g., zero/few-shot learning, protein structure prediction) could us more precise articulation. 
    \item \textbf{Comparative Analysis of LLMs}: The paper could provide a detailed comparative analysis of different LLMs, including their strengths, weaknesses, and limitations, to help readers understand the trade-offs involved in choosing one model over another for a specific bioinformatics task.his section could use more explanations, claims, arguments which wouold lead to more opportunities to cite.
    \item  \textbf{Referencnes: }None were mentioned in the paper.
    \item \textbf{Story line:} the abstract was the only thing that had a flow which resembles the story element. The reminder of the paper was obviously not enough to judge about the logical flow of the document.
\end{itemize}
\end{document}
