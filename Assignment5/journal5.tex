\documentclass{article}
\usepackage{cite}
\usepackage[english]{babel}
\usepackage[letterpaper,top=2cm,bottom=2cm,left=3cm,right=3cm,marginparwidth=1.75cm]{geometry}
\usepackage{amsmath}
\usepackage{graphicx}
\usepackage{tikz}
\usepackage{url}
\usepackage{hyperref}
\title{Journal5}
\author{Raja Kantheti}
\begin{document}
\maketitle

\section*{Summary of an Academic Misconduct Case: }
The story can be found at this \href{https://en.wikipedia.org/wiki/Chris_Spence_(educator)}{link}.


Christopher M. Spence is a retired Canadian educator, author, and former Canadian football player, who served as Director of Education for both the Hamilton-Wentworth District School Board and the Toronto District School Board. Born in England to Jamaican parents, he moved to Canada at the age of three and completed his education in Windsor, Ontario. Spence earned a B.A. in criminology from Simon Fraser University, played as a running back in the CFL for the BC Lions until an Achilles tendon injury ended his career, and obtained his B.Ed., M.Ed., and Ed.D. degrees from York University and the University of Toronto.

His career faced a downfall due to plagiarism scandals. In January 2013, he publicly apologized for plagiarizing content in an op-ed for the Toronto Star and resigned as Director of Education after further instances of plagiarism in his writings were discovered. An investigation revealed that his 1996 Ed.D. dissertation contained numerous plagiarized passages, leading the University of Toronto to rescind his degree in June 2017. Subsequently, the Ontario College of Teachers revoked his teaching license in December 2016.

In 2018, Spence lost an appeal to retain his Ed.D., with an appeals tribunal describing the plagiarism as a serious offense. However, he successfully appealed the revocation of his teaching license on the grounds that his mental health issues, including depression and suicidal thoughts stemming from personal challenges, were not properly considered. He intends to seek judicial review regarding the Ed.D. revocation.
\section*{Progress on the survey paper:}
I successfully set up the simulator for all the branch prediction techniques, CPU and the riscv architecture.
I have also started to understand the Addvanced branch prediction techniques and how they are implemented in the \emph{gem5} simulator.
I would like to have more papers on some of the techniques like Multiperspective Perceptron and Tage-SC-L. That should cover the literature review for the survey paper. 

I also started to take notes on everything while scanning and not spending more than 3 minutes on each paper. I should be done with scanning all the papers in a week and get into critical reads on everything. In the mean time I am focusing more on how I want to organize the paper. 
I have some ideas for now, but I would like to wait a few more days before I confirm to anything. In addition to those, I would really like to a different story than what I have right now, so thiking on that more. 
Zotero has beenn really helpful to me in taking notes and organizing the plethora of papers. I should really take into consideration, \copyright Litmaps, for brain mapping. 
\section*{ChatGpt Theme/Metaphor:}
\subsection*{ Branch Prediction: The Ultimate Game of CPU Hide-and-Seek}
Branch prediction techniques are likened to a game of hide-and-seek where the CPU is the seeker trying to find the correct execution path among many hiding spots (branch paths). The survey acts as the referee, keeping track of all the clever hiding spots and strategies to ensure the CPU finds its way efficiently, all while avoiding the pitfalls of mispredictions that can lead to confusion and delays.
\section*{Theme/Metaphor I came up with: }
\subsection*{Surfing the Waves of RISC-V: Catching the Right Branch}
Branch prediction techniques are like surfers trying to ride the perfect wave on a chaotic sea of execution paths. Some surfers (techniques) are better at catching waves (predicting branches) than others, and this survey helps readers identify which surfers (techniques) can ride the waves of RISC-V out-of-order processors most gracefully, ensuring a smooth ride instead of wiping out on mispredictions.

\section*{Alternate Story 1:}
We have suitable branch prediction techniques depending on the type of workload. So, if can have a standard to segregate workloads and if we can test the availbale branch techniques ffor each category, we can 
gt more results. 
\subsection*{Advantages: }
\begin{itemize}
    \item Increased Scope.
    \item A more interesting approach for a survey paper.
    \item More results.
\end{itemize}
\subsection*{Disadvantages: }
\begin{itemize}
    \item Complication over segregating the workloads.
    \item Need more Computation power to be able to get the results over sevral iteration if ever we can segregate the workloads.
\end{itemize}
\paragraph*{Note: }I was only able to think of one \emph{alternative} story for now with this new refined survey topic. 
\end{document}
