\documentclass{article}
\usepackage{cite} % Add the cite package for citations

% Language setting
% Replace `english' with e.g. `spanish' to change the document language
\usepackage[english]{babel}

% Set page size and margins
% Replace `letterpaper' with `a4paper' for UK/EU standard size
\usepackage[letterpaper,top=2cm,bottom=2cm,left=3cm,right=3cm,marginparwidth=1.75cm]{geometry}

% Useful packages
\usepackage{amsmath}
\usepackage{graphicx}
\usepackage[colorlinks=true, allcolors=blue]{hyperref}

\title{Journal, Week-3}
\author{Raja Kantheti}

\begin{document}
\maketitle

\section{Introduction}
Five Characteristics of a good survey paper.
\begin{enumerate}
    \item Relevance with current research aspects
    \item Logic in comparision
    \item Higher Magnitude in number of comparative entities
    \item Unbiased experimental set up 
    \item Analysis and Insights.
\end{enumerate}

\section{My Reasearch Idea}
I would like to research how a multiple fetching and decode
units inn processoor pipeline can either improve or eliminate
the brnach prediction in superscalar micro controllers. My goal for a paper
in this course is to gather information about contemporary branch prediction techiniques used in 
different processors and also get a quantification how noel this approach is so that I might pursue a Ph.D on this. 
\\
(yes, I typed it without thinking.)

\section{Inner Child Voice}
\subsection{Theory: }
Imagine a man who just does everything you tell him to do one thing at a time. Now he sometimes has a problem in figuring out what to do next.
For example if you are teaching him how to make some chicken, and if you told him, "Put the chicken if the oven is preheated, otherwise wait for the oven", Sometimes the man guesses
the oven is preheated and puts the chicken in, if the oven isn't ready he will take out the chicken and wait. But putting in the chicken and taking it out again is a waste of time unless he already knows
that he can just wait. So he an assistant who would have the next step always ready for the man so that he wouldn't waste time with guessing and correcting.
\subsection{Illustration: }




\section{5 Papers and their notes:}

\subsection{Paper 1:}
\emph{Title: }"A Survey of Machine Learning Applied to Computer Architecture Design"\cite{penneySurveyMachineLearning2019}\\
\emph{Notes: }
Story: We needed innovative design strategies
because traditional design processors were slow and less efficient
but machine learning applications were not fully explored
so they surveyed ML techniques in architecture design
finally uncovering new potentials for performance enhancements.

\subsection{Paper 2:}
\emph{Title: }"Performance Analysis of Big.LITTLE System with Various Branch Prediction Schemes"\cite{rodriguesPerformanceAnalysisBigLITTLE2021}\\
\emph{Notes: }
Analysts needed to evaluate ARM's Big.LITTLE ARchitecture
because performance varied widely with workload types 
but branch prediction impacts were unclear 
so they tested multiple prediction schemes 
finally identifying optimal configurations for different scenarios.

\subsection{Paper 3:}
\emph{Title: }"A Survey of Techniques for Dynamic Branch Prediction"\cite{mittalSurveyTechniquesDynamic2019a}\\
\emph{Notes: }
Computers needed effective branch prediction techniques 
because pipeline stalls were degrading processor performance 
but simple predictors were insufficient 
so the paper revies and categorizes advanced methods 
finally providing a thorough understanding of current techniques.

\subsection{Paper 4:}
\emph{Title: }"A Survey of Deep Learning Techniques for Dynamic Branch Prediction"\cite{josephSurveyDeepLearning2021}\\
\emph{Notes: }
We needed to improve branch prediction accuracy
because the traditional methods were causing bottlenecks in teh aspect of performance
so the paper reviews deep learning techniques 
finally identifying the most promising approaches for future research.

\paragraph*{}
The highest citied paper  in the given is " A survey of Deep Learning TEchniques for Dynamic Branch Prediction" with 81 citations.
The lowest cited paper in the given is "Performance Analysis of Big.LITTLE System with Various Branch Prediction Schemes" with 12 citations.
I think this is because the deep learning techniques are more popular and have more potential for future research, whre as the Big.LITTLE system has a narrow scope with a specific architecture and may not be as widely applicable.

\section{Potential Paper Story: }
\begin{enumerate}
    \item Correlation between branch prediction techniques and Branch mispredictions. \paragraph{Story:} We need to understand how branch misprediction can be correlated to several existing techniques to identify the gaps.
    \item A survey on the current branch prediction techniques used in contemporary processors. \paragraph{Story: } We need to understand the current state of branch prediction techniques to understand why they work.
    \item A comparative study on the performance of different branch prediction techniques. \paragraph{Story: } We need to understand the performance of different branch prediction techniques to identify the most effective ones.
    \item A study on the impact of multiple fetch and decode units on branch prediction and performance. \paragraph{Story: } We need to understand how a puipeline change can make changes in branch mispredictions.
\end{enumerate}


\bibliographystyle{plain}
\bibliography{Journal3}
\end{document}